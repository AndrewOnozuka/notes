\documentclass[a4paper,10pt]{article}
\usepackage[utf8]{inputenc}
\usepackage[margin=0.7in]{geometry}
\usepackage{amsmath}
\usepackage{algorithm}
\usepackage{booktabs}
\usepackage[noend]{algpseudocode}
\usepackage{tabularx}
\usepackage{graphicx}
\usepackage{amssymb}
\usepackage{enumitem}
\usepackage{comment}
\usepackage{color}
\usepackage{xcolor}
\newcommand{\handout}[5]{
   \renewcommand{\thepage}{}
   \noindent
   \begin{center}
   \framebox{
      \vbox{
    \hbox to 6in { {\bf CSE 141: Introduction to Computer Architecture}
         \hfill #2 }
       \vspace{4mm}
       \hbox to 6in { {\Large \hfill #1  \hfill} }
       \vspace{2mm}
       \hbox to 6in { {\emph{#3} \hfill #4 \emph{#5}} }
      }
   }
   \end{center}
   \vspace*{2mm}
   
   Name: \underline{\hspace{4cm}}
  
   PID: \underline{\hspace{4 cm}}
   
   Email: \underline{\hspace{4 cm}}
}

\newcommand{\homework}[5]{\handout{Homework #1}{#2}{Instructor: #3}{{\bf Due on:} #4}{#5}}

\title{Bibliography management: BibTeX}
\author{Share\LaTeX}

\begin{document}
\homework{4}{Spring 2024}{Jishen Zhao} {June 11, 2024}{(66 points)}

\subsection*{Instructions}
\begin{itemize}
\item The homework must be submitted to Gradescope by 11:59pm. {Anything later is a late submission}
\item Handwritten or typed responses are accepted. 
\item All responses must be neat and legible. Illegible answers will result in zero points.
\item Provide details on how to reach a solution. An answer without explanation gets no credit. Clearly state all assumptions.



\end{itemize}

\vspace*{2mm}


\begin{enumerate}

\item (14 points) \textbf{Main Memory} 

Following is a system and DIMM description.
\begin{itemize}
    \item 2 memory channels
    \item 2 DIMMS per channel
    \item Rank: 1
    \item Chips per rank: 8
    \item Column Size: 8 bits
    \item Banks per Chip: 16
    \item Rows per Bank: 65,536
    \item Columns per Bank: 1,024
    \item 8-byte Bus
\end{itemize}

Answer the following questions based on the given information. \textbf{Specify units wherever applicable and write how you derived your answer.} e.g., ${2^2}\times{2^1}=2^3$
\begin{enumerate}
    \item (2 points) How much physical memory is in a single chip?
    \item (2 points) How much physical memory is in a single DIMM?
    \item (2 points) How much physical memory is in the whole system?
    \item (2 points) What is the minimum number of physical address bits needed to address this much memory?
    \item (6 points) Which portions of the physical address bits are used for column, channel, bank, row, and byte in a bus? Assume that the total number of bits for this question excludes 1 bit to select DIMM. Include only the fields given above. Also, use the row interleaving scheme.
    \newline

\end{enumerate}
\\

\item (12 points) \textbf{Virtual Addressing}. 
This question explores the different parameters in a virtual address scheme.
Consider a machine with a 40-bit virtual address, 16KB page size, with 4 bytes per page table entry. Assume no protection bits. \textbf{Write how you derived your answer.} e.g., $\frac{2^2}{2^1}=2^1$

Determine the following:
\begin{enumerate}
    \item (4 points) What is the number of pages in the virtual address space?

    \item (4 points) What is the maximum size of addressable physical memory in the system?

    \item (4 points) What would be the size of a one-level page table?
\end{enumerate}

\item (24 points) \textbf{Cache Coherence Protocols}

In this question we will explore different cache coherence protocols, for a processor with 4 cores. You are given a sequence of memory accesses for variable x, and you are to determine the state of the cache line containing x, in each core for the sequence of accesses.

Each subdivision uses a different cache coherence protocol, and this question explores how different protocols affect cache behavior. This question only considers the state of the L1 cache.

\textbf{3.A VI Protocol (8 Points)}

Assume for this part that we use the VI protocol for cache coherence. Fill out the table as specified at the beginning of the question.

\begin{center}
\begin{tabular}{|c|c|c|c|c|c|}
\hline
\textbf{Core} & \textbf{Request} & \textbf{C0 Cache} & \textbf{C1 Cache} & \textbf{C2 Cache} & \textbf{C3 Cache} \\
 & \textbf{Type} & \textbf{Line State} & \textbf{Line State} & \textbf{Line State} & \textbf{Line State} \\
\hline
0 & Read $x$ &   &   &   &   \\
\hline
1 & Read $x$ &   &   &   &   \\
\hline
2 & Read $x$ &   &   &   &   \\
\hline
3 & Write $x$ &   &   &   &   \\
\hline
1 & Read $x$ &   &   &   &   \\
\hline
\end{tabular}
\end{center}

\textbf{3.B MSI Protocol (8 Points)}

Assume for this part that we use the MSI protocol for cache coherence. Fill out the table as specified at the beginning of the question.

\begin{center}
\begin{tabular}{|c|c|c|c|c|c|}
\hline
\textbf{Core} & \textbf{Request} & \textbf{C0 Cache} & \textbf{C1 Cache} & \textbf{C2 Cache} & \textbf{C3 Cache} \\
 & \textbf{Type} & \textbf{Line State} & \textbf{Line State} & \textbf{Line State} & \textbf{Line State} \\
\hline
0 & Read $x$ &   &   &   &   \\
\hline
1 & Read $x$ &   &   &   &   \\
\hline
2 & Read $x$ &   &   &   &   \\
\hline
3 & Write $x$ &   &   &   &   \\
\hline
1 & Read $x$ &   &   &   &   \\
\hline
\end{tabular}
\end{center}

\textbf{3.C MESI Protocol (8 Points)}

Assume for this part that we use the MESI protocol for cache coherence. Fill out the table as specified at the beginning of the question.

\begin{center}
\begin{tabular}{|c|c|c|c|c|c|}
\hline
\textbf{Core} & \textbf{Request} & \textbf{C0 Cache} & \textbf{C1 Cache} & \textbf{C2 Cache} & \textbf{C3 Cache} \\
 & \textbf{Type} & \textbf{Line State} & \textbf{Line State} & \textbf{Line State} & \textbf{Line State} \\
\hline
0 & Read $x$ &   &   &   &   \\
\hline
1 & Read $x$ &   &   &   &   \\
\hline
2 & Read $x$ &   &   &   &   \\
\hline
3 & Write $x$ &   &   &   &   \\
\hline
1 & Read $x$ &   &   &   &   \\
\hline
\end{tabular}
\end{center}

\item (16 points) \textbf{Cache Coherence with Memory Subsystem}

In this question, we will explore an implementation of the MESI protocol, in a more complex system, with 4 CPU cores (C0, C1, C2, and C3). Each of those cores has a private 32KB L1 Cache, with 64 byte cacheline size, is 2-way set-associative, and uses write back. All 4 cores share a 512KB L2 cache which has 64 byte cacheline size, and is 8-way set associative.

Similar to question 3, you are given a sequence of memory accesses, and you are to write down the state of each core’s chacheline, for each access.
Note that the initial state of the cache is given to you, with the first memory access performed, which means this is not a cold start.

\begin{center}
\begin{tabular}{|c|c|c|c|c|c|c|}
\hline
\textbf{Core} & \textbf{Request} & \textbf{Address} & \textbf{C0 L1} & \textbf{C1 L1} & \textbf{C2 L1} & \textbf{C3 L1} \\
 & \textbf{Type} &  & \textbf{State} & \textbf{State} & \textbf{State} & \textbf{State} \\
\hline
0 & Load & 0x00ffabc0 & E & - & - & - \\
\hline
0 & Store & 0x00ffabcc &   &   &   &   \\
\hline
1 & Load & 0x00ffabd4 &   &   &   &   \\
\hline
1 & Store & 0x00ffabd8 &   &   &   &   \\
\hline
1 & Load & 0x00afabc0 &   &   &   &   \\
\hline
2 & Load & 0x00afabc8 &   &   &   &   \\
\hline
1 & Load & 0x00bfabf0 &   &   &   &   \\
\hline
0 & Load & 0x00ffabc0 &   &   &   &   \\
\hline
\end{tabular}
\end{center}



\end{enumerate}

\end{document}

