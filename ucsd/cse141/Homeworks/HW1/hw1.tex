\documentclass[a4paper,10pt]{article}
\usepackage[utf8]{inputenc}
\usepackage[margin=0.7in]{geometry}
\usepackage{amsmath}
\usepackage{algorithm}
\usepackage[noend]{algpseudocode}
\usepackage{tabularx}
\usepackage{graphicx}
\usepackage{amssymb}
\usepackage{enumitem}
\usepackage{comment}
\usepackage{color}
\usepackage{xcolor}
\newcommand{\handout}[5]{
   \renewcommand{\thepage}{}
   \noindent
   \begin{center}
   \framebox{
      \vbox{
    \hbox to 6in { {\bf CSE 141: Introduction to Computer Architecture}
         \hfill #2 }
       \vspace{4mm}
       \hbox to 6in { {\Large \hfill #1  \hfill} }
       \vspace{2mm}
       \hbox to 6in { {\emph{#3} \hfill #4 \emph{#5}} }
      }
   }
   \end{center}
   \vspace*{2mm}
   
   Name: \underline{\hspace{4cm}}
  
   PID: \underline{\hspace{4 cm}}
   
   Email: \underline{\hspace{4 cm}}
}

\newcommand{\homework}[5]{\handout{Homework #1}{#2}{Instructor: #3}{{\bf Due on:} #4}{#5}}

\title{Bibliography management: BibTeX}
\author{Share\LaTeX}

\begin{document}
\homework{1}{Spring 2024}{Jishen Zhao} {Apr 26, 2024}{(50 points)}

\subsection*{Instructions}
\begin{itemize}
\item The homework must be submitted to Gradescope by 11:59pm. { Anything later is a late submission}
\item Handwritten or typed responses are accepted. 
\item All responses must be neat and legible. Illegible answers will result in zero points.
\item Provide details on how to reach a solution. An answer without explanation gets no credit. Clearly state all assumptions.

\item This homework covers performance and ISA.

\end{itemize}

\vspace*{2mm}


\begin{enumerate}
\begin{comment}
    QUESTION 1
\end{comment}
\item (15 points) \textbf{Speedup and Amdahl's Law}. 

\begin{enumerate}
    \item \textbf{\textit{(5 points)}} A program runs 20 billion instructions on a 2Ghz processor, achieving a CPI of 1.5. You are then introduced to an improved ISA, that results in a speedup of 1.2 when recompiling and executing the same program on the same processor. Given that when you recompile with the new ISA you have a reduced instruction count of 15 billion instructions, calculate the CPI of the newly compiled code.
    
    \item \textbf{\textit{(5 points)}} There exists a program for which $\frac{1}{3}$ of the instructions are inherently serial, and the rest are paralellizable. If you expect a maximum speedup of 2.4 over single core execution using an n-core multi-core processor, find the value of n.
    
    \item \textbf{\textit{(5 points)}} You are given a program that has code with the following characteristics:
    \begin{itemize}
        \item 10\% of the code is inherently serial.
        \item 55\% of the code can be executed in parallel only with the CPU.
        \item 35\% of the code can be executed in parallel with either the GPU or CPU.
    \end{itemize}
    Your task is to choose a machine that will result in the fastest execution of the program. You have two machines to choose from; 
    
    A) the first has an 8-core multi-core CPU and a 4-core GPU.
    
    B) the second has a 6-core multi-core CPU and an 8-core GPU. 
    
    Given these two machines, which would you choose for the given program?

    HINT: Using GPU cores may not always result in the fastest execution of the program.
    
\end{enumerate}

\vspace{\baselineskip}
\begin{comment}
    QUESTION 2
\end{comment}

\item (11 points) \textbf{ISA Design Trade-offs}

Given that you have a processor with 32 general purpose registers and an instruction length of 12 bits, state if each of the following scenarios are possible. If not, state the minimum number of instruction bits you would actually need to make the scenario possible.

    \emph{Scenario 1} \textbf{\textit{(3 points)}} -
    
    30 instructions that use two register operands.\\
    
    \emph{Scenario 2} \textbf{\textit{(4 points)}} -
    
    2 instructions with one register operand and an immediate value. The immediate value will be a two's compliment binary number that can represent a value of at least 32.\\
    
    \emph{Scenario 3} \textbf{\textit{(4 points)}} -
    
    All three of the following instruction formats:
    \begin{itemize}
        \item 3 instructions that use two register operands.
        \item 30 instructions that use one register operands.
        \item 45 instructions that use no register operands.
    \end{itemize}

    HINT: First consider how many unique instructions (unique opcodes and operands) you can have with each instruction format. With this information you should be able to deduce if the instruction length is enough to encode all these unique instructions. \\

\vspace{\baselineskip}

\begin{comment}
    QUESTION 3
\end{comment}

\item (8 points) \textbf{C to MIPS} 

An incomplete set of MIPS instructions along with a short explanation of their operations are given below. Complete the given assembly instructions such that it result in the C statement: $$B[g] = A[f + 4] - A[f]$$ 
Assume variables \textit{f} and \textit{g} are assigned to \$s0 and \$s1 respectively and base addresses of arrays \textit{A} and \textit{B} are in registers \$s6 and \$s7, respectively. 

To store any intermediate values, you may use temporary registers (\$t0 - \$t9).

NOTE: In this scenario, the memory is byte-addressable, and each index of arrays A and B corresponds to 4 bytes of memory.

\begin{verbatim}
    sll $t0, $s0, 2         # $t0 = f * 4
    add $t0, $s6, $t0       # $t0 = &A[f]
    sll $t1, $s1, 2         # $t1 = g * 4
    add $t1, $s7, $t1       # $t1 = &B[g]
    lw  $s0, 0($t0)         # f = A[f]
\end{verbatim}
    
\vspace{\baselineskip}
\begin{comment}
    QUESTION 4
\end{comment}

\item (8 points) \textbf{Processor Optimization}

The table below shows the distribution of each instruction type on a processor running with a 2 GHz clock frequency.
\begin{center}
\begin{tabular}{ c c c c c}
Instr &	Proportion	& CPI (baseline) & CPI (opt \#1) & CPI (opt \#2) \\ 
Load/Store & 30\% & 5 & 2 & 3 \\ 
Branch & 25\% & 2 & 1 & 2 \\  
Mul/Div	& 10\% & 8 & 7 & 5 \\ 
Other & 35\% & 1 & 1 & 1 \\
Total & 100\%
\end{tabular}
\end{center}

\begin{enumerate}
    \item \textbf{A Baseline CPI \textit{(2 points)}}
    What is the average CPI of the baseline processor?
    
    \item \textbf{Optimization \textit{(6 points)}}
    With some optimization, and the support of some extra logic, the microarchitecture can decrease the load and store CPI to 2 and some other CPI decreases as shown in the table (option \#1). Alternatively, the extra logic that can be spent can be used to optimize the mul/div logic to support CPI = 5 and some other CPI decreases as shown in the table (option \#2). What is the CPI speedup of each optimization relative to the baseline design? Which optimization is better to improve performance? (Use the tables provided above).
    
\end{enumerate}

\vspace{\baselineskip}

\begin{comment}
    QUESTION 5
\end{comment}

\item \textbf{Architectures and Instruction Sets} (8 points)

\begin{enumerate}
    \item \textbf{\textit{(4 points)}} Compare and contrast CISC with RISC. What are the trade-offs? Why choose one over the other? (compare at least two aspects)

    \item \textbf{\textit{(4 points)}} Which of the following properties and metrics are dependent on the ISA chosen (RISC vs CISC)? Why?

    \begin{itemize}
        \item Instruction latency.
        \item Die area.
        \item Energy efficiency (think about energy per instruction).
        \item Code length.
    \end{itemize}

\end{enumerate}

\end{enumerate}

%\framebox(475, 600){}

\end{document}